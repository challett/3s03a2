\documentclass[]{article}

% Imported Packages
%------------------------------------------------------------------------------
\usepackage{amssymb}
\usepackage{amstext}
\usepackage{amsthm}
\usepackage{amsmath}
\usepackage{enumerate}
\usepackage{fancyhdr}
\usepackage[margin=1in]{geometry}
\usepackage{graphicx}
\usepackage{extarrows}
\usepackage{setspace}
\usepackage{listings}
\usepackage{color}

\definecolor{dkgreen}{rgb}{0,0.6,0}
\definecolor{gray}{rgb}{0.5,0.5,0.5}
\definecolor{mauve}{rgb}{0.58,0,0.82}

\lstset{frame=tb,
  language=Java,
  aboveskip=3mm,
  belowskip=3mm,
  showstringspaces=false,
  columns=flexible,
  basicstyle={\small\ttfamily},
  numbers=none,
  numberstyle=\tiny\color{gray},
  keywordstyle=\color{blue},
  commentstyle=\color{dkgreen},
  stringstyle=\color{mauve},
  breaklines=true,
  breakatwhitespace=true,
  tabsize=3
}
%------------------------------------------------------------------------------

% Header and Footer
%------------------------------------------------------------------------------
\pagestyle{plain}  
\renewcommand\headrulewidth{0.4pt}                                      
\renewcommand\footrulewidth{0.4pt}                                    
%------------------------------------------------------------------------------

% Title Details
%------------------------------------------------------------------------------
\title{Assignment 2 Test Report}
\author{Connor Hallett}
\date{March 23 2015}                               
%------------------------------------------------------------------------------

% Document
%------------------------------------------------------------------------------
\begin{document}

\maketitle	

\section{General Information}
\label{sec:introduction}

	\subsection{Summary}
	The program being tested is designed to promt the user to enter the name of a
text file.  IF the file cannot be opened, the program should catch any
FileNotFoundExceptions, notify the user, and ask for another input until three
failures occur.  If a proper file name is entered, the program will read the
content of the file, count the number of occurences of the character A, D, and W
in the third field of the record.  The program will output a summary on the
number of occurences of these characters on each line in a file named
``output.txt''. Stress testing, error handling, and black box testing were used
in this test.

	\subsection{Environment}
	The program was
provided by Dr. Ridha Khedri of McMaster University department of Computing and
Software.  The testing will be preformed by Connor Hallett, a third year student
of software engineering and management at McMaster University.

\section{Test Results and Findings}

\subsection{Error Handling - Wrong Filename}
% Begin SubSection
	\subsubsection{Validation}
	% Begin SubSubSection
	This test found that the program did not accept the correct number of attempts
	to enter a proper filename.  After three attempts, the user was still prompted
	to enter a filename.  After the fourth filename was entered, the program
	terminated gracefully with a message.  A summary of the error and steps to fix
	it can be found on the next page.
	% End SubSubSection
	
	\subsubsection{Verification}
	% Begin SubSubSection
	Below is the output from the program when faced with the user entering an
	incorrect filename until they are no longer allowed.
	% End SubSubSection
	
	\begin{lstlisting}
	Welcome to The Record Processor
	-------------------------------
	Please input the input file name: tt.txt
	Sorry, cannot open the file for reading!
	Please input the input file name: tt.txt
	Sorry, cannot open the file for reading!
	Please input the input file name: tt.txt
	Sorry, cannot open the file for reading!
	Please input the input file name: tt.txt
	Sorry, cannot open the file for reading!
	It is over!
	\end{lstlisting}


	\newpage
	
% End SubSection

\subsection{Stress Testing - Empty File}
% Begin SubSection
	\subsubsection{Validation}
	% Begin SubSubSection
	The program passed the test as expected.
	% End SubSubSection
	
	\subsubsection{Verification}
	% Begin SubSubSection
	Input for this test was test2.txt.  Below is the output from the program
	\begin{lstlisting}
	output.txt
	
	
	
	\end{lstlisting}
	% End SubSubSection
	
	\newpage
	
% End SubSection

\subsection{Error Handling - Third entry too large}
% Begin SubSection
	\subsubsection{Validation}
	% Begin SubSubSection
	Input for this test was test5.txt.  This test did not raise any errors. 
	However, the output in output.txt was incorrect.  The numbers shown for the counts of A are much smaller than
	expected.
	% End SubSubSection
	
	\subsubsection{Verification}
	% Begin SubSubSection
	\begin{lstlisting}
	output.txt
	2, 1, 5, 0,      8, 1, 5, 0
	3, 1, 5, 0,      9, 1, 5, 0
	4, 1, 5, 0,      10, 1, 5, 0
	5, 1, 5, 0,      11, 0, 0, 0
	6, 1, 5, 0,      12, 0, 0, 0
	For the above 6 lines, number of A: 8 **** number of D: 40 **** number of V: 0
	\end{lstlisting}
	\begin{lstlisting}
	expected output.txt
	1, 1048577, 5, 1,      6, 1048577, 5, 1
	2, 1048577, 5, 1,      7, 1048577, 5, 1
	3, 1048577, 5, 1,      8, 1048577, 5, 1
	4, 1048577, 5, 1,      9, 1048577, 5, 1
	5, 1048577, 5, 1,      10, 0, 0, 0
	6, 1048577, 5, 1,      11, 0, 0, 0
	For the above 6 lines, the number of A: 6291462 **** the number of D: 30 **** the number of W: 6
	7, 1048577, 5, 1,      12, 0, 0, 0
	8, 1048577, 5, 1,      13, 0, 0, 0
	9, 1048577, 5, 1,      14, 0, 0, 0
	10, 1048577, 5, 1,      15, 0, 0, 0
	\end{lstlisting}
	% End SubSubSection
	
	\newpage
	
% End SubSection

\subsection{Stress Testing - Third entry maximum size}
% Begin SubSection
	\subsubsection{Validation}
	% Begin SubSubSection
	The file ``test6.txt'' was used for this test.  This test showed that overflow
	occured even when the input was inside of the range to be expected.  A summary
	of this defect and steps to fix it can be found on the next page.
	% End SubSubSection
	
	\subsubsection{Verification}
	% Begin SubSubSection
	\begin{lstlisting}
	output.txt
	2, -6, 5, 0,      8, -6, 5, 0
	3, -6, 5, 0,      9, -6, 5, 0
	4, -6, 5, 0,      10, -6, 5, 0
	5, -6, 5, 0,      11, 0, 0, 0
	6, -6, 5, 0,      12, 0, 0, 0
	For the above 6 lines, number of A: -48 **** number of D: 40 **** number of V: 0

	\end{lstlisting}
	\begin{lstlisting}
	expected output.txt
	1, 1048570, 5, 1,      6, 1048570, 5, 1
	2, 1048570, 5, 1,      7, 1048570, 5, 1
	3, 1048570, 5, 1,      8, 1048570, 5, 1
	4, 1048570, 5, 1,      9, 1048570, 5, 1
	5, 1048570, 5, 1,      10, 0, 0, 0
	6, 1048570, 5, 1,      11, 0, 0, 0
	For the above 6 lines, the number of A: 6291420 **** the number of D: 30 **** the number of W: 6
	7, 1048570, 5, 1,      12, 0, 0, 0
	8, 1048570, 5, 1,      13, 0, 0, 0
	9, 1048570, 5, 1,      14, 0, 0, 0
	10, 1048570, 5, 1,      15, 0, 0, 0
	\end{lstlisting}
	% End SubSubSection
	
	\newpage
	
% End SubSection

\subsection{Blackbox Testing - 1}
% Begin SubSection
	\subsubsection{Validation}
	% Begin SubSubSection
	The input file test1.txt and test7.txt were used for these blackbox tests.  The
	tests revealed that the program did not properly count the number of W's in the
	third fields of the input files.  It also revealed that multiple lines of the
	input file were being skipped, and that the sum in the six line summaries was
	incorrect.  Overviews and remedies for these issues can be found on the next
	few pages.
	% End SubSubSection
	
	\subsubsection{Verification}
	% Begin SubSubSection
	\begin{lstlisting}
	test1.txt
	output.txt
	2, 1, 6, 0,      8, 0, 9, 0
	3, 3, 2, 0,      9, 14, 16, 0
	4, 14, 10, 0,      10, 18, 0, 0
	5, 0, 6, 0,      11, 0, 12, 0
	6, 13, 12, 0,      12, 16, 0, 0
	For the above 6 lines, number of A: 79 **** number of D: 73 **** number of V: 0
	13, 17, 4, 0,      19, 4, 8, 0
	14, 15, 14, 0,      20, 15, 6, 0
	15, 17, 18, 0,      21, 1, 18, 0
	16, 19, 18, 0,      22, 5, 9, 0
	17, 14, 19, 0,      23, 5, 12, 0
	18, 3, 18, 0,      24, 8, 17, 0
	For the above 6 lines, number of A: 123 **** number of D: 161 **** number of V: 0
	25, 10, 4, 0,      31, 0, 0, 0
	\end{lstlisting}
	\begin{lstlisting}
	expected output.txt
	1, 18, 14, 15,      6, 4, 4, 1
	2, 1, 6, 15,      7, 0, 9, 10
	3, 3, 2, 19,      8, 14, 16, 14
	4, 14, 10, 11,      9, 18, 0, 3
	5, 0, 6, 17,      10, 0, 12, 6
	6, 13, 12, 12,      11, 16, 0, 2
	For the above 6 lines, the number of A: 49 **** the number of D: 50 **** the number of W: 89
	7, 4, 4, 1,      12, 17, 4, 8
	8, 0, 9, 10,      13, 15, 14, 9
	9, 14, 16, 14,      14, 17, 18, 3
	10, 18, 0, 3,      15, 19, 18, 13
	11, 0, 12, 6,      16, 14, 19, 1
	12, 16, 0, 2,      17, 3, 18, 18
	For the above 6 lines, the number of A: 52 **** the number of D: 41 **** the number of W: 36
	13, 17, 4, 8,      18, 4, 8, 9
	14, 15, 14, 9,      19, 15, 6, 18
	15, 17, 18, 3,      20, 1, 18, 2
	16, 19, 18, 13,      21, 5, 9, 11
	17, 14, 19, 1,      22, 5, 12, 8
	18, 3, 18, 18,      23, 8, 17, 16
	For the above 6 lines, the number of A: 85 **** the number of D: 91 **** the number of W: 52
	19, 4, 8, 9,      24, 10, 4, 13
	20, 15, 6, 18,      25, 0, 0, 0
	21, 1, 18, 2,      26, 0, 0, 0
	22, 5, 9, 11,      27, 0, 0, 0
	23, 5, 12, 8,      28, 0, 0, 0
	24, 8, 17, 16,      29, 0, 0, 0
	For the above 6 lines, the number of A: 38 **** the number of D: 70 **** the number of W: 64
	25, 10, 4, 13,      30, 0, 0, 0
	\end{lstlisting}
	
	\begin{lstlisting}
	test7.txt
	output.txt
	2, 18, 18, 0,      8, 12, 14, 0
	3, 7, 17, 0,      9, 18, 2, 0
	4, 16, 16, 0,      10, 18, 12, 0
	5, 16, 13, 0,      11, 5, 8, 0
	6, 15, 11, 0,      12, 6, 13, 0
	For the above 6 lines, number of A: 131 **** number of D: 124 **** number of V: 0
	13, 8, 10, 0,      19, 6, 5, 0
	14, 17, 15, 0,      20, 11, 10, 0
	15, 14, 11, 0,      21, 10, 14, 0
	16, 3, 8, 0,      22, 9, 14, 0
	17, 6, 3, 0,      23, 11, 17, 0
	18, 17, 2, 0,      24, 14, 10, 0
	For the above 6 lines, number of A: 126 **** number of D: 119 **** number of V: 0
	25, 10, 18, 0,      31, 10, 7, 0
	26, 4, 1, 0,      32, 13, 9, 0
	27, 9, 7, 0,      33, 5, 18, 0
	28, 8, 4, 0,      34, 11, 17, 0
	29, 16, 11, 0,      35, 18, 1, 0
	30, 3, 3, 0,      36, 0, 3, 0
	For the above 6 lines, number of A: 107 **** number of D: 99 **** number of V: 0
	37, 9, 19, 0,      43, 18, 1, 0
	38, 12, 2, 0,      44, 14, 6, 0
	39, 12, 14, 0,      45, 0, 0, 0
	40, 1, 9, 0,      46, 0, 0, 0
	41, 5, 17, 0,      47, 0, 0, 0
	42, 5, 0, 0,      48, 0, 0, 0
	For the above 6 lines, number of A: 76 **** number of D: 68 **** number of V: 0

	\end{lstlisting}
	\begin{lstlisting}
	expected output.txt
	1, 1, 11, 14,      6, 19, 12, 11
	2, 18, 18, 17,      7, 12, 14, 2
	3, 7, 17, 18,      8, 18, 2, 14
	4, 16, 16, 7,      9, 18, 12, 12
	5, 16, 13, 9,      10, 5, 8, 17
	6, 15, 11, 13,      11, 6, 13, 6
	For the above 6 lines, the number of A: 73 **** the number of D: 86 **** the number of W: 78
	7, 19, 12, 11,      12, 8, 10, 14
	8, 12, 14, 2,      13, 17, 15, 15
	9, 18, 2, 14,      14, 14, 11, 10
	10, 18, 12, 12,      15, 3, 8, 6
	11, 5, 8, 17,      16, 6, 3, 15
	12, 6, 13, 6,      17, 17, 2, 15
	For the above 6 lines, the number of A: 78 **** the number of D: 61 **** the number of W: 62
	13, 8, 10, 14,      18, 6, 5, 11
	14, 17, 15, 15,      19, 11, 10, 11
	15, 14, 11, 10,      20, 10, 14, 17
	16, 3, 8, 6,      21, 9, 14, 6
	17, 6, 3, 15,      22, 11, 17, 16
	18, 17, 2, 15,      23, 14, 10, 3
	For the above 6 lines, the number of A: 65 **** the number of D: 49 **** the number of W: 75
	19, 6, 5, 11,      24, 10, 18, 7
	20, 11, 10, 11,      25, 4, 1, 9
	21, 10, 14, 17,      26, 9, 7, 0
	22, 9, 14, 6,      27, 8, 4, 17
	23, 11, 17, 16,      28, 16, 11, 5
	24, 14, 10, 3,      29, 3, 3, 5
	For the above 6 lines, the number of A: 61 **** the number of D: 70 **** the number of W: 64
	25, 10, 18, 7,      30, 10, 7, 0
	26, 4, 1, 9,      31, 13, 9, 9
	27, 9, 7, 0,      32, 5, 18, 17
	28, 8, 4, 17,      33, 11, 17, 3
	29, 16, 11, 5,      34, 18, 1, 4
	30, 3, 3, 5,      35, 0, 3, 10
	For the above 6 lines, the number of A: 50 **** the number of D: 44 **** the number of W: 43
	31, 10, 7, 0,      36, 9, 19, 7
	32, 13, 9, 9,      37, 12, 2, 18
	33, 5, 18, 17,      38, 12, 14, 2
	34, 11, 17, 3,      39, 1, 9, 3
	35, 18, 1, 4,      40, 5, 17, 5
	36, 0, 3, 10,      41, 5, 0, 7
	For the above 6 lines, the number of A: 57 **** the number of D: 55 **** the number of W: 43
	37, 9, 19, 7,      42, 18, 1, 9
	38, 12, 2, 18,      43, 14, 6, 11
	39, 12, 14, 2,      44, 0, 0, 0
	40, 1, 9, 3,      45, 0, 0, 0
	41, 5, 17, 5,      46, 0, 0, 0
	42, 5, 0, 7,      47, 0, 0, 0
	For the above 6 lines, the number of A: 44 **** the number of D: 61 **** the number of W: 42
	43, 18, 1, 9,      48, 0, 0, 0
	44, 14, 6, 11,      49, 0, 0, 0
	\end{lstlisting}
	% End SubSubSection
	
	\newpage
	
% End SubSection

\subsection{Error Handling - Too many lines in input file}
% Begin SubSection
	\subsubsection{Validation}
	% Begin SubSubSection
	The file test3.txt was used for this test.  This test revealed that the program
	did not gracefully reject files that were too long.  Instead, an exception was
	returned in the console to the user.
	% End SubSubSection
	
	\subsubsection{Verification}
	% Begin SubSubSection
	\begin{lstlisting}
	Welcome to The Record Processor
	-------------------------------
	Please input the input file name: test3.txt
	Exception in thread "main" java.lang.ArrayIndexOutOfBoundsException: 50
		at RecordReader.readThirdFieldsFromFile(RecordReader.java:47)
		at RecordProcessor.main(RecordProcessor.java:136)
	\end{lstlisting}
	% End SubSubSection
	
	\newpage
	
% End SubSection

\subsection{Stress Testing - File with maximum number of lines as input}
% Begin SubSection
	\subsubsection{Control}
	% Begin SubSubSection
	The file test4.txt was used for this test.  This test revealed that the program
	does not accept as input a file that contains 50 lines.  The maximum number of
	lines that the program will accept in its current state is 49.  A summary of
	this defect and how to fix it can be found on the next page.
	% End SubSubSection
	
	\subsubsection{Inputs}
	% Begin SubSubSection
	\begin{lstlisting}
	Welcome to The Record Processor
	-------------------------------
	Please input the input file name: test4.txt
	Exception in thread "main" java.lang.ArrayIndexOutOfBoundsException: 50
		at RecordProcessor.process(RecordProcessor.java:41)
		at RecordProcessor.main(RecordProcessor.java:140)
	\end{lstlisting}
	% End SubSubSection
	
	\newpage
	
% End SubSection

\subsection{Analysis Summary}
% Begin SubSection

	\subsubsection{Capabilities}
	% Begin SubSubSection
	In its current state, this program is only capable of producing acceptable
	output on files containing zero lines.  These tests have shown multiple
	discrepencies between the expected and actual outputs, and have shown numerous
	defects in the program.
	% End SubSection
	
	\subsubsection{Deficiencies}
	% Begin SubSubSection
	The program falls short on producing adequate output when the user attempts to
	process any file of length greater than one.  Due to the program skipping
	lines, miscalculating sums, and reporting zeros for all instances of `'W'', the
	programs output is not to be trusted in its current state.
	% End SubSection
	
	\subsubsection{Risks}
	% Begin SubSubSection
	Should this program be placed in production, anything that uses its output will
	likely fail.  Since the program does not live up to its specification, and
	produces extremely poor results, the consequences of using this software could
	be catastrophic.
	% End SubSection
	
	\subsubsection{Reccommendations}
	% Begin SubSubSection
	Before this program is placed in production.  I advise that the steps detailed
	in the numerous problem documents are followed.  Once these steps have been put
	in place.  It is advised that the original test plan is run again before any
	additional tests take place.
	% End SubSection
	

\end{document}
%------------------------------------------------------------------------------