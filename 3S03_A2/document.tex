\documentclass[]{article}
% \begin{enumerate}
%   \item	The input file shall contain no more than 50 lines.  Each line shall be
%   formed by a record where the fields are delineated by commas.  The record
%   shall contain at least 4 fields and no more than 10 fields.
%   \item The third field on any given line of the input file will always contain
%   less than $2^{20}$ characters.
%   \item 
%\end{enumerate}

% Imported Packages
%------------------------------------------------------------------------------
\usepackage{amssymb}
\usepackage{amstext}
\usepackage{amsthm}
\usepackage{amsmath}
\usepackage{enumerate}
\usepackage{fancyhdr}
\usepackage[margin=1in]{geometry}
\usepackage{graphicx}
\usepackage{extarrows}
\usepackage{setspace}
\usepackage{listings}
\usepackage{color}

\definecolor{dkgreen}{rgb}{0,0.6,0}
\definecolor{gray}{rgb}{0.5,0.5,0.5}
\definecolor{mauve}{rgb}{0.58,0,0.82}

\lstset{frame=tb,
  language=Java,
  aboveskip=3mm,
  belowskip=3mm,
  showstringspaces=false,
  columns=flexible,
  basicstyle={\small\ttfamily},
  numbers=none,
  numberstyle=\tiny\color{gray},
  keywordstyle=\color{blue},
  commentstyle=\color{dkgreen},
  stringstyle=\color{mauve},
  breaklines=true,
  breakatwhitespace=true,
  tabsize=3
}
%------------------------------------------------------------------------------

% Header and Footer
%------------------------------------------------------------------------------
\pagestyle{plain}  
\renewcommand\headrulewidth{0.4pt}                                      
\renewcommand\footrulewidth{0.4pt}                                    
%------------------------------------------------------------------------------

% Title Details
%------------------------------------------------------------------------------
\title{SE3S03 Assignment 2 Test Plan}
\author{Connor Hallett 1158083}
\date{}                               
%------------------------------------------------------------------------------

% Document
%------------------------------------------------------------------------------
\begin{document}

\maketitle	

\section{GENERAL INFORMATION}
\label{sec:introduction}
% Begin Section

\subsection{Summary}
\label{sub:purpose}
% Begin SubSection
The program being tested is designed to promt the user to enter the name of a
text file.  IF the file cannot be opened, the program should catch any
FileNotFoundExceptions, notify the user, and ask for another input until three
failures occur.  If a proper file name is entered, the program will read the
content of the file, count the number of occurences of the character A, D, and W
in the third field of the record.  The program will output a summary on the
number of occurences of these characters on each line in a file named
``output.txt''.  This test plan will test the correctness of this program based
on inputs and outputs, and the performance of the system under extreme
conditions and boundary cases.
% End SubSection

\subsection{Environment and Pretest Background}
\label{sub:scope}
% Begin SubSection
This is the first iteration of testing for this program.  The program was
provided by Dr. Ridha Khedri of McMaster University department of Computing and
Software.  The testing will be preformed by Connor Hallett, a third year student
of software engineering and management at McMaster University.
% End SubSection

\subsection{Test Objectives}
\label{sub:definitions_acronyms_and_abbreviations}
% Begin SubSection
This testing iteration aims to detect as many defects in the program as
possible, and to provide an estimate of the number of defects remaining through
the use of error seeding. The performance of the system under boundary
conditions will be examined through the use of stress testing, and the
correctness of the program output will be examined with whitebox testing.
% End SubSection

\subsection{Expected Defect Rates}
\label{sub:references}
% Begin SubSection
Due to the size and nature of the project, it is forcasted that there will be
roughly thirty errors per one hundred lines of code in the program.  A more
formal estimate will be provided after the completion of this testing iteration.
% End SubSection

% \subsection{Overview}
% \label{sub:overview}
% % Begin SubSection
% \begin{enumerate}[a)]
% 	\item Describe what the rest of the SRS contains
% 	\item Explain how the SRS is organized
% \end{enumerate}
% % End SubSection

% End Section

\section{PLAN}
\label{sec:overall_description}
% Begin Section

\subsection{Software Description}
\label{sub:product_perspective}
	% Begin SubSection
	The program will ask the user for the name of a file as input.  If the input
	from the user raises  a \verb!FileNotFoundException!, the program will notify
	the user and ask for another input, up to a maximum of three attempts. Inputs to the
	program will be given in a plaintext document containing at most fifty lines containing a record of four to ten fields delineated by a comma.
	The program will read the contents of the file, and will produce a file named
	``output.txt'' containing lines of the following format when the line i+6
	exists:
	\begin{center}
	\verb!i,_n(i,A),_n(i,D),_n(i,W),_____i+6,_n(i+6,A),_n(i+6,D),_n(i+6,W)!\
	\end{center}
	where \_ denotes a space character.  If the line indexed with i+6 does not
	exist, the line in output should be as follows:
	\begin{center}
	\verb!_i,_n(i,A),_n(i,D),_n(i,W),______i + 6,_0,_0,_0!
	\end{center}
	Additionally, every six lines of the output file should separated a summary of
	the previous six lines in the format:
	\begin{center}
	\verb!For the above 6 lines, number of A:_nA_****_number of D:_nD_!
	\verb!****_number of W: _nW!
	\end{center}

% End SubSection

\subsection{Test Team}
\label{sub:product_functions}
% Begin SubSection
The test team will Consist of Connor Hallett and \_\_\_\_\_\_\_\_\_.  Connor
will preform the primary testing, and his findings will be used as seeds in the tests
conducted by Jeremy.  This use of error seeding will allow Connor to estimate
the number of defects remaining in the program after testing.
% End SubSection

\subsection{Milestones}
\label{sub:user_characteristics}
% Begin SubSection
\begin{enumerate}[1)]
	\item The entire test plan shall be completed and executed for Monday March 23,
	2015
	\item A report on the discovered defects and results of the test plan will be
	completed for Monday March 23, 2015
\end{enumerate}
% End SubSection

\subsection{Budget}
\label{sub:constraints}
% Begin SubSection
	A strict budget of \$0.00 CAD will be imposed on this project.  Excluding
	resources paid for in advance such as printer ink, paper, and electricity.  The
	tester executing the plan will work for free, and will not contract out any
	work.
% End SubSection

% 
% \subsection{Assumptions and Dependencies}
% \label{sub:assumptions_and_dependencies}
% % Begin SubSection
% \begin{enumerate}[a)]
% 	\item List each of the factors that affect the requirements stated in the SRS
% 	\item These factors are not design constraints on the software but are, rather, any changes to them that can affect the requirements in the SRS
% 	\begin{itemize}
% 		\item \textbf{Example}: An assumption may be that a specific operating system will be available on the hardware designated for the software product. If, in fact, the operating system is not available, the SRS would then have to change accordingly.
% 	\end{itemize}
% \end{enumerate}
% % End SubSection
% 
% \subsection{Apportioning of Requirements}
% \label{sub:apportioning_of_requirements}
% % Begin SubSection
% \begin{enumerate}[a)]
% 	\item Identify requirements that may be delayed until future versions of the system
% \end{enumerate}
% % End SubSection

% End Section

\section{SPECIFICATIONS AND EVALUATIONS}
\label{sec:functional_requirements}
% Begin Section

\subsection{Specifications}
\label{sub:user_characteristics}
% Begin SubSection

\subsubsection{Business Functions}
\label{sub:user_characteristics}
% Begin SubSubSection
\begin{enumerate}
  \item The program will take the name of a file as input from the user and will
  check if the name provided exists.  Should the name not exist, the user will
  be notified and will be allowed to attempt to enter a name a maximum of two
  additional times.
  \item When an appropriate filename has been entered by the user, the program
  will create the file ``output.txt'' containing information on the number of
  appearances of the characters A, D, and W in the third field of every line in
  the input file in the aforementioned format (See 2.1).
  \item Each six lines of the output file will be seperated by a line
  containing a total number of characters reported in the previous six lines.
\end{enumerate}
% End SubSubSection

\subsubsection{Structural Functions}
\label{sub:user_characteristics}
% Begin SubSubSection

\begin{enumerate}
  \item RecordReader.java
  \begin{lstlisting}
  class RecordReader {
  ...
  public static String[] readThirdFieldsFromFile(String inputFileName)
    throws FileNotFoundException, IOException {
    ...
    while (record != null) {
      ...
    }
    ...
    return thirdFields;

  }
  \end{lstlisting} 

  \item RecordProcessor.java
% Begin SubSubSubSection
\begin{lstlisting}
class RecordProcessor
...
public static void process(String[] thirdFields) throws IOException {
	...
	while (lineNum <= RecordReader.MAX_LINE_NUM && !endReached) {
      ...
    }
    ...
	if (i + SUBSECTION_LENGTH < lineNum) {
	  ...
	}
	...
	if ((i + 1) % (2 * SUBSECTION_LENGTH) == SUBSECTION_LENGTH) {
	  ...
	  if (k < chars.length - 1) {
	  ...
        for (int k = 0; k < chars.length; k++) {
          ...
          if (k < chars.length - 1) {
            ...
          }
        ...
	    }
	  ...
	  }
...
}

 public static void main(String args[]) throws IOException {
 	...
	while ((!inputValid) && count < 4) {
	  try {
		...
	  }
	  catch (IOException e) {
	    if (count >= 4){System.out.println("It is over!");};
	  }
	}
	...
}
\end{lstlisting}
\end{enumerate}
% End SubSubSubSection

% End SubSubSection

\subsubsection{Test/Function Relationships}
\label{sub:user_characteristics}
% Begin SubSubSection

\begin{enumerate}
  \item RecordReader.java \break
	In order to reach the entire structure of RecordReader, we will focus on
	\verb!readThirdFieldsFromFile!.  The String input will be
	chosen as follows: \begin{enumerate}
  					\item ``test1.txt'' : A proper text file that contains 25 lines in the
  					expected format.  This is expected to reach the \verb!while(record /=!
  					\verb!null)! loop, execute its contents, and return an array containing
  					the third fields of the file.
  					\item ``test2.txt'' : A proper text file that contains nothing.  This
  					input is expected to bypass the while loop and return \verb!null!.  
  					\item ``test3.txt'' : A proper text file that contain 51 lines in the
  					expected format.  It is expected that this function will only return an
  					array error from attempting to access an element in an array that is out
  					of the arrays predetermined size.
  					\item ``test4.txt'' : A proper text file that contains 50 lines in the
  					expected format.  It is expected that this function will return an array
  					containing all 50 third fields of the input file.
  					\item ''tt.txt'' : A filename that does not exist.  This is expected to
  					throw \verb!FileNotFoundException!
					\end{enumerate}
  \item RecordProcessor.Java \break
	
\end{enumerate}

% End SubSubSection

% End SubSection
\subsection{Methods and Constraints}
% Begin SubSection
	\subsubsection{Methodology}
	\label{sub:user_characteristics}
	% Begin SubSubSection
	The majority of the testing in this testing iteration will be performed
	manually inside of Eclipse.  If any changes are made throughout testing,
	regression testing will be used to verify that all previously passed tests
	still pass after the change is made.
	% End SubSubSection
	
	\subsubsection{Test Tools}
	% Begin SubSubSection
	Test cases will be generated using the scripting language Python.
	Tests will be executed manually through Eclipse.
	% End SubSubSection
	
	\subsubsection{Extent}
	% Begin SubSubSection
	This test plan aims to have total coverage of the program.
	% End SubSubSection
	
	\subsubsection{Data Recording}
	% Begin SubSubSection
	Data will be recorded manually into the Test Summary Report throughout testing.
	Contents of the ``output.txt'' file will be manually inspected for correctness.
	% End SubSubSection
	
	\subsubsection{Constrains}
	% Begin SubSubSection
	The main constraints on this test plan is the lack of personel available for
	test purposes, the manual nature of execution of the test cases, and the
	inspection required of the output file.
	% End SubSubSection
% End SubSection

\subsection{Evaluation}
% Begin SubSection
	\subsubsection{Criteria}
	% Begin SubSubSection
	The test plan aims to cover as many possible inputs as possible on the
	functions defined in the program.  This means utilizing empty files, null
	strings, files with lengths longer than the maximum acceptable length, and
	files that contain rectores with the third field containing more than $2^{20}$
	characters.
	% End SubSubSection
	
% End SubSection

% End Section

\section{Non-Functional Requirements}
\label{sec:non-functional_requirements}
% Begin Section
\subsection{Look and Feel Requirements}
\label{sub:look_and_feel_requirements}
% Begin SubSection

\subsubsection{Appearance Requirements}
\label{ssub:appearance_requirements}
% Begin SubSubSection
\begin{enumerate}[{LF}1. ]
	\item 
\end{enumerate}
% End SubSubSection

\subsubsection{Style Requirements}
\label{ssub:style_requirements}
% Begin SubSubSection
\begin{enumerate}[{LF}1. ]
	\item 
\end{enumerate}
% End SubSubSection

% End SubSection

\subsection{Usability and Humanity Requirements}
\label{sub:usability_and_humanity_requirements}
% Begin SubSection

\subsubsection{Ease of Use Requirements}
\label{ssub:ease_of_use_requirements}
% Begin SubSubSection
\begin{enumerate}[{UH}1. ]
	\item 
\end{enumerate}
% End SubSubSection

\subsubsection{Personalization and Internationalization Requirements}
\label{ssub:personalization_and_internationalization_requirements}
% Begin SubSubSection
\begin{enumerate}[{UH}1. ]
	\item 
\end{enumerate}
% End SubSubSection

\subsubsection{Learning Requirements}
\label{ssub:learning_requirements}
% Begin SubSubSection
\begin{enumerate}[{UH}1. ]
	\item 
\end{enumerate}
% End SubSubSection

\subsubsection{Understandability and Politeness Requirements}
\label{ssub:understandability_and_politeness_requirements}
% Begin SubSubSection
\begin{enumerate}[{UH}1. ]
	\item 
\end{enumerate}
% End SubSubSection

\subsubsection{Accessibility Requirements}
\label{ssub:accessibility_requirements}
% Begin SubSubSection
\begin{enumerate}[{UH}1. ]
	\item 
\end{enumerate}
% End SubSubSection

% End SubSection

\subsection{Performance Requirements}
\label{sub:performance_requirements}
% Begin SubSection

\subsubsection{Speed and Latency Requirements}
\label{ssub:speed_and_latency_requirements}
% Begin SubSubSection
\begin{enumerate}[{PR}1. ]
	\item 
\end{enumerate}
% End SubSubSection

\subsubsection{Safety-Critical Requirements}
\label{ssub:safety_critical_requirements}
% Begin SubSubSection
\begin{enumerate}[{PR}1. ]
	\item 
\end{enumerate}
% End SubSubSection

\subsubsection{Precision or Accuracy Requirements}
\label{ssub:precision_or_accuracy_requirements}
% Begin SubSubSection
\begin{enumerate}[{PR}1. ]
	\item 
\end{enumerate}
% End SubSubSection

\subsubsection{Reliability and Availability Requirements}
\label{ssub:reliability_and_availability_requirements}
% Begin SubSubSection
\begin{enumerate}[{PR}1. ]
	\item 
\end{enumerate}
% End SubSubSection

\subsubsection{Robustness or Fault-Tolerance Requirements}
\label{ssub:robustness_or_fault_tolerance_requirements}
% Begin SubSubSection
\begin{enumerate}[{PR}1. ]
	\item 
\end{enumerate}
% End SubSubSection

\subsubsection{Capacity Requirements}
\label{ssub:capacity_requirements}
% Begin SubSubSection
\begin{enumerate}[{PR}1. ]
	\item 
\end{enumerate}
% End SubSubSection

\subsubsection{Scalability or Extensibility Requirements}
\label{ssub:scalability_or_extensibility_requirements}
% Begin SubSubSection
\begin{enumerate}[{PR}1. ]
	\item 
\end{enumerate}
% End SubSubSection

\subsubsection{Longevity Requirements}
\label{ssub:longevity_requirements}
% Begin SubSubSection
\begin{enumerate}[{PR}1. ]
	\item 
\end{enumerate}
% End SubSubSection

% End SubSection

\subsection{Operational and Environmental Requirements}
\label{sub:operational_and_environmental_requirements}
% Begin SubSection

\subsubsection{Expected Physical Environment}
\label{ssub:expected_physical_environment}
% Begin SubSubSection
\begin{enumerate}[{OE}1. ]
	\item 
\end{enumerate}
% End SubSubSection

\subsubsection{Requirements for Interfacing with Adjacent Systems}
\label{ssub:requirements_for_interfacing_with_adjacent_systems}
% Begin SubSubSection
\begin{enumerate}[{OE}1. ]
	\item 
\end{enumerate}
% End SubSubSection

\subsubsection{Productization Requirements}
\label{ssub:productization_requirements}
% Begin SubSubSection
\begin{enumerate}[{OE}1. ]
	\item 
\end{enumerate}
% End SubSubSection

\subsubsection{Release Requirements}
\label{ssub:release_requirements}
% Begin SubSubSection
\begin{enumerate}[{OE}1. ]
	\item 
\end{enumerate}
% End SubSubSection

% End SubSection

\subsection{Maintainability and Support Requirements}
\label{sub:maintainability_and_support_requirements}
% Begin SubSection

\subsubsection{Maintenance Requirements}
\label{ssub:maintenance_requirements}
% Begin SubSubSection
\begin{enumerate}[{MS}1. ]
	\item 
\end{enumerate}
% End SubSubSection

\subsubsection{Supportability Requirements}
\label{ssub:supportability_requirements}
% Begin SubSubSection
\begin{enumerate}[{MS}1. ]
	\item 
\end{enumerate}
% End SubSubSection

\subsubsection{Adaptability Requirements}
\label{ssub:adaptability_requirements}
% Begin SubSubSection
\begin{enumerate}[{MS}1. ]
	\item 
\end{enumerate}
% End SubSubSection

% End SubSection

\subsection{Security Requirements}
\label{sub:security_requirements}
% Begin SubSection

\subsubsection{Access Requirements}
\label{ssub:access_requirements}
% Begin SubSubSection
\begin{enumerate}[{SR}1. ]
	\item Users should have access to login to their account and be able to request or offer a taxi 24 hours a day, seven days a week.
	\item Unregistered users should have access to creating an account 24 hours a day, seven days a week.
	\item Users should be able to modify their account details while not actively searching for somebody to share a taxi with.
\end{enumerate}
% End SubSubSection

\subsubsection{Integrity Requirements}
\label{ssub:integrity_requirements}
% Begin SubSubSection
\begin{enumerate}[{SR}1. ]
	\item All changes to information in the database should be checked for proper format before saving.
	\item User login credentials will be verified before any modifications to their data.
\end{enumerate}
% End SubSubSection

\subsubsection{Privacy Requirements}
\label{ssub:privacy_requirements}
% Begin SubSubSection
\begin{enumerate}[{SR}1. ]
	\item All information about registered users in the database should be protected from outside access.
	\item All details held about a user will be unavailable to any other user unless matched for a potential taxi share.
	\item All chat messages sent between users should be encripted during transmission.
	\item No exact passwords will be stored in the database.
\end{enumerate}
% End SubSubSection

\subsubsection{Audit Requirements}
\label{ssub:audit_requirements}
% Begin SubSubSection
\begin{enumerate}[{SR}1. ]
	\item N/A
\end{enumerate}
% End SubSubSection

\subsubsection{Immunity Requirements}
\label{ssub:immunity_requirements}
% Begin SubSubSection
\begin{enumerate}[{SR}1. ]
	\item Upon detection of a corrupt file, the file in question will be deleted and reinstalled to maintain the integrity of the system.
\end{enumerate}
% End SubSubSection

% End SubSection

\subsection{Cultural and Political Requirements}
\label{sub:cultural_and_political_requirements}
% Begin SubSection

\subsubsection{Cultural Requirements}
\label{ssub:cultural_requirements}
% Begin SubSubSection
\begin{enumerate}[{CP}1. ]
	\item 
\end{enumerate}
% End SubSubSection

\subsubsection{Political Requirements}
\label{ssub:political_requirements}
% Begin SubSubSection
\begin{enumerate}[{CP}1. ]
	\item 
\end{enumerate}
% End SubSubSection

% End SubSection

\subsection{Legal Requirements}
\label{sub:legal_requirements}
% Begin SubSection

\subsubsection{Compliance Requirements}
\label{ssub:compliance_requirements}
% Begin SubSubSection
\begin{enumerate}[{LR}1. ]
	\item 
\end{enumerate}
% End SubSubSection

\subsubsection{Standards Requirements}
\label{ssub:standards_requirements}
% Begin SubSubSection
\begin{enumerate}[{LR}1. ]
	\item 
\end{enumerate}
% End SubSubSection

% End SubSection

% End Section

\appendix
\section{Division of Labour}
\label{sec:division_of_labour}
% Begin Section
Include a Division of Labour sheet which indicates the contributions of each team member. This sheet must be signed by all team members.
% End Section

\newpage
\section*{IMPORTANT NOTES}
\begin{itemize}
	\item Be sure to include all sections of the template in your document regardless whether you have something to write for each or not
	\begin{itemize}
		\item If you do not have anything to write in a section, indicate this by the \emph{N/A}, \emph{void}, \emph{none}, etc.
	\end{itemize}
	\item Uniquely number each of your requirements for easy identification and cross-referencing
	\item Highlight terms that are defined in Section~1.3 (\textbf{Definitions, Acronyms, and Abbreviations}) with \textbf{bold}, \emph{italic} or \underline{underline}
	\item For Deliverable 1, please highlight, in some fashion, all (you may have more than one) creative and innovative features. Your creative and innovative features will generally be described in Section~2.2 (\textbf{Product Functions}), but it will depend on the type of creative or innovative features you are including.
\end{itemize}


\end{document}
%------------------------------------------------------------------------------